\documentclass[11pt]{article}
\usepackage[english]{babel}
\usepackage{acl2014}
\usepackage{times}
\usepackage{url}
\usepackage{latexsym}

\begin{document}
\title{Linear Text Regression on Movie Subtitles}
\author{Brian Bechtel, Jonathan Stites\\
       University of Texas at Austin}
\date{}

\maketitle

\begin{abstract}
\noindent
Plot specific features may be useful in predicting box office revenue of films.
In this paper we present a method for analyzing movie subtitles as such a predictive
feature using Linear Text Regression. We compare the performance of subtitles against
using movie reviews and find that subtitles do not perform as well in predictive power.
We discuss alternative features that reflect plot should extensions of this work be
done in the future.
\end{abstract}

\section{Introduction}
Text regression is a a useful tool to predict real world values from textual inputs.
In 2010, Joshi et al. used text regression on movie reviews to predict box office revenue.
In their paper they analyzed the performance of using reviews as predictor vs using meta
data alone (such as budget, number of screens, release date, etc.). They found that text
from reviews could both substitute for and improve over a strong meta data based approach.

A limitation in the methodology of Joshi et al. is that they used features that are only
available late into film's production, such as pre-release movie reviews. A pre-release
review requires a cut of the film, which means much, if not all, of the filming has
finished. Thus, the procedure described by Joshi et al. is limited in that a prediction
can only be made late into the production cycle of a film. At such a late point,
significant investments have already been made in the film and thus a prediction of
revenue is of less use for financial planning.

The main goal of this paper was to try and identify prediction features that occur much
earlier in the film making process. We hypothesized that features that captured a film's
plot directly might be useful for predicting revenue. If such features prove useful as
predictors of revenue, then as a consequence prediction can be done much earlier in the
film making process, greatly improving the utility of the model.

We initially wanted to use the film's script as such a feature, but found that getting
enough script data was challenging as there ended up not being any great sources of
move script data. From the sources we were able to identify we were only able to
obtain about $140$ movie scripts of any use, which was not nearly enough data to perform
proper regression on. In retrospect, this makes sense as film scripts are considered
intellectual property, and thus the scripts are not widely available to the public.

As an alternative to scripts, we decided to use the subtitles for the films instead. The
majority of text in a script is the characters' dialogue, and thus since subtitles contain
all the dialogue of the film, we felt that they were a reasonable proxy for scripts.
Scripts do contain other textual information not in subtitles, such as describing a scene,
which makes subtitles not a perfect substitute for scripts.

In the following sections we describe our methodology for analyzing the subtitle text
and making predictions of box office revenue. Our findings indicate that subtitles do
not perform as well as movie reviews in predicting revenue, though there is some small
predictive value. In general our attempt to find predictive features that can be used
earlier in the film making process as compared to those presented in Joshi et al. was 
unsuccessful. We do discuss alternative features that can be tested in our sections on
future work. Perhaps some of these features may end up performing as well as movie reviews
in predicting box office revenue.

\section{Prediction Model}
\subsection{Task Definition}
Precisely define the problem you are addressing (i.e. formally specify the inputs and
outputs). Elaborate on why this is an interesting and important problem. 

\subsection{Algorithm Definition}
Describe in reasonable detail the algorithm you are using to address this problem. A
psuedocode description of the algorithm you are using is frequently useful. Trace through
a concrete example, showing how your algorithm processes this example. The example should
be complex enough to illustrate all of the important aspects of the problem but simple
enough to be easily understood. If possible, an intuitively meaningful example is better
than one with meaningless symbols.

\section{Experimental Evaluation}
\subsection{Methodology}
What are criteria you are using to evaluate your method? What specific hypotheses does
your experiment test? Describe the experimental methodology that you used. What are the
dependent and independent variables? What is the training/test data that was used, and why
is it realistic or interesting? Exactly what performance data did you collect and how are
you presenting and analyzing it? Comparisons to competing methods that address the same
problem are particularly useful.

\subsection{Results}
Present the quantitative results of your experiments. Graphical data presentation such as
graphs and histograms are frequently better than tables. What are the basic differences
revealed in the data. Are they statistically significant?

\subsection{Discussion}
Is your hypothesis supported? What conclusions do the results support about the strengths
and weaknesses of your method compared to other methods? How can the results be explained
in terms of the underlying properties of the algorithm and/or the data.

\section{Related Work}
Answer the following questions for each piece of related work that addresses the same or
a similar problem. What is their problem and method? How is your problem and method
different? Why is your problem and method better?

\section{Future Work}
What are the major shortcomings of your current method? For each shortcoming, propose
additions or enhancements that would help overcome it

\section{Conclusion}
Briefly summarize the important results and conclusions presented in the paper. What are
the most important points illustrated by your work? How will your results improve future
research and applications in the area?

\begin{thebibliography}{9}

\bibitem{Joshi2010}
  Mahesh Joshi, Dipanjan Das, Kevin Gimpel, Noah A. Smith,
  Movie Reviews and Revenues: An Experiment in Text Regression,
  Proceedings of NAACL-HLT,
  2010.
  
\bibitem{Bitvai2015}
    Zsolt Bitvai, Trevor Cohn,
    Non-Linear Text Regression with a Deep Convolutional Neural Network,
    Proceedings of the 53rd Annual Meeting of the Association for Computational Linguistics
    and the 7th International Joint Conference on Natural Language Processing,
    2015.
    
\bibitem{JoshiData}
  Mahesh Joshi, Dipanjan Das, Kevin Gimpel, Noah A. Smith,
  "Movie\$ Data",
  Web,
  Accessed 08 April 2017,
  https://www.cs.cmu.edu/~ark/movie\$-data/

\bibitem{IMSDb}
    "The Internet Movie Script Database",
    Web,
    Accessed 08 April 2017,
    http://www.imsdb.com/disclaimer/
  
\end{thebibliography}
\end{document}
%%% Local Variables: 
%%% mode: latex
%%% TeX-master: t
%%% End: 

